\section{Introduction to Android Wear}

Android Wear is a version of Google's Android operating system tailored for
wearables. At the time of writing, Android Wear is only available for
smartwatches, but one can assume that it will be extended to other wearable
devices in the forseeable future.\\
Android Wear was announced on March 18, 2014 and the first publicly available
devices were released on June 25, 2014 at Google I/O. These devices were the
"LG G Watch" and the "Samsung Gear Live" (not to be confused with the
"Samsung Galaxy Gear" which is not an Android Wear device.)

Upon release, Android Wear shipped running "Android 4.4W" (API 20) or "Android
KitKat" for wearables and could communicate with mobile devices running
"Android 4.3" (API 18) or higher. At the time of writing, the latest Android
Wear revision is Android 5.0.2 (API 21), more commonly known as "Android
Lollipop".

Android Wear is relatively restricted in what it can do on it's own. It can run
applications and services but may need a connection to a mobile device to carry
out certain tasks. For example, smartwatches have no internet connection and so
will need to use the mobile device as a proxy if web-related content is needed.
GPS is only available on select smartwatches such as the "Moto 360" and so the
mobile device may again be useed as proxy if necessary.
It is also recommended to use the mobile device for long running, or
computationally intensive tasks so as not to drain the wearable device's
battery.

Since Android Wear uses the same base version of Android as phones and tablets,
applications are written in Java and configured using XML just like on mobile.
Most layouts and views work without any changes, however some input fields such
as \texttt{EditText}s are useless since the screen is too small to comfortably
fit a keyboard.\\
Instead of requesting input from a user in the form of text fields, it is
favourable to use speech recognition to allow the user to input textual data.
Some layouts and views are specific to Android Wear such as
\texttt{WatchViewStub}, which allows the developer to select different layouts
for square and circular screens.

At the time of writing there are multiple circular screen devices on the market
such as the first circular smartwatch, the "Moto 360" from Motorola and the "G
Watch R" from LG. By default, layouts will work on both rectangular and
circular screens, but text/parts of the User Interface (UI) may be clipped
depending on radius of the screen and the pixel density.

\subsection{Development Environment}
Android Studio is the recommended Integrated Development Environment (IDE) to
use when developing for Android Wear, although with the recent switch to the
Gradle build system, it is still possible to build in other IDE's such as
Eclipse, or even use no IDE at all and use the command-line Gradle wrapper to
build the application.

For Android Wear to be able to communicate with a mobile device, the mobile
device needs to have the official Android Wear companion app
\footnote{https://play.google.com/store/apps/details?id=com.google.android.wearable.app\&hl=en}
installed and running. The companion app is necessary for pairing the two
devices together in a secure way.

Android Wear system images can be downloaded using Android's SDK Manager and run
in the emulator just like the regular versions of Android. Debugging using the
emulator is swift and easy, but has it's limitations. If the application needs
to communicate with the mobile device, the emulator must be set up to recognize
it. To allow the emulator to communicate with another emulator or physical
device, the ports on the machine must be setup using the command:
\begin{lstlisting}[language=Bash]
$ adb -d forward tcp:5601 tcp:5601
\end{lstlisting}

When debugging with physical devices, there are three ways in which the Android
Debugging Bridge (ADB) can access the device:

\begin{enumerate}

\item USB\\
    This is the most straight-forward method for debugging and is a simple as
    connecting the device to the development computer and ensuring the device is
    visible to ADB and has the correct permissions to communicate with it.
\item Bluetooth\\
    Debugging over Bluetooth is a bit trickier to set up, and much slower in
    terms of transferring data. The following commands are used to setup the
    necessary ports:
    \begin{lstlisting}
    $ adb forward tcp:4444 localabstract:/adb-hub
    $ adb connect localhost:4444
    \end{lstlisting}
\item WiFi\\
    Debugging over WiFi is only possible with mobile devices as the wearables
    do not have WiFi capabilities. However to setup a mobile device for
    debugging over WiFi, the following commands are necessary:
    \begin{lstlisting}
    $ adb tcpip 5555
    $ adb connect #.#.#.#:5555
    \end{lstlisting}
    where \texttt{\#.\#.\#.\#} is the ip address of the mobile device
\end{enumerate}
