\section{Design}

This chapter aims to give an overview of the architecture and structure of the
main components of the project.

\section{Overview}
As previously mentioned, the project is made up of three main parts:
\begin{enumerate}
\item Research Project\\
This project aims to explore the Android Wear APIs and display real world data
obtained from the smartwatch's sensors. Since this project's sole purpose is to
display experimental data, it's user interface is kept minimal and
straight-forward.
\item Wearable Student Application\\
This is the main project of the three and is an application for both
smartwatches and mobile/tablet devices. It's goal is to provide students with
a simple and easy to use app to help them keep track of their lectures and
results whilst also demonstrating some of the newer APIs available in Android
"Lollipop"
\item Calendar Watchface\\
This project provides an example of how to use the new Watchface APIs released
by Google in January 2014. Previously, the APIs were undocumented but a working
prototype was built in the Research Project and was converted to the new
documented APIs in this project.

\end{enumerate}

\section{Package Overview}
% explain gms packages
TODO

\section{Project Architecture}
This section details the internal architecture of each of the three projects
outlined above.

\subsection{Research Project}
This project contains both a smartwatch application and a mobile/tablet
application. Three modules are used and compiled using the Gradle build system:
\begin{enumerate}
\item wear\\
the module containing the wearable application source code, resources etc.
\item mobile\\
the same as the \texttt{wear} module but for the mobile application's code and
resources.
\item common\\
the module containing the shared code, such as utility libraries and model
objects likely to be used in both applications. This module also contains
images, layouts and other resources that are made available to the other two
modules.
\end{enumerate}

\subsection{Student Application}
The Student Application has a similar architecture to the research project with
three main modules: one for the mobile application, one for the wear application
and one for shared code and resources.

\subsection{Watchface}
The watchface consists of a mobile application and a wear application. The
mobile application is empty as all Android Wear applications need a mobile app
so that they can be installed on the mobile and synced over to the watch.
The mobile application could however be useful, to allow the user to change some
of the watchface settings via a settings menu on the mobile rather than trying
to change the settings on the watch.

\section{User Interface}

% Use cases, flow diagrams etc.?
