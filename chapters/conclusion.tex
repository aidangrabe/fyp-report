In this chapter, the accomplishments of the project will be reviewed and
discussed. It will also discuss the potential for future developments.

\section{Overview}

This project met all of its original requirements and provides a detailed view
of the brand new technology that is Android Wear. At the time of writing there
were few quality, extensive, detailed guides on using Android Wear. This
dissertation aims to provide such a guide by explaining the use of the APIs and
demonstrating them by providing a sample application that students may also find
useful.

The project offers the following features:
\begin{enumerate}
\item A detailed description of the Android Wear communication APIs
\item Examples of how to create notifications that work on both a handheld
    device but also on a wearable.
\item Examples of how to poll sensors and use their data for displaying to 
    a user or performing calculations based on their values.
\item Performing actions based on 3d gestures performed by the user.
\item A useable application that helps students stay on top of their studies
    by helping them track their lectures, result progress, college news and
    track their tasks.
\item A few wearable games that were designed with a small screen in mind.
\item An example of how multiple wearables can be used together to create a 
    multiplayer game of Snake.
\item A watchface that can show the user their events for the day at a quick
    glance.
\item An example of how to build a custom watchface which adheres to the Google
    watchface guidelines and that works on both circular and rectangular
    watches.
\end{enumerate}

The project demonstrated the use of many different tools and technologies:
\begin{enumerate}
\item the applications were written using Java and the Android and Android
    Wear APIs.
\item Android Studio was used for the IDE.
\item Git was used for version control.
\item Gradle was used as the build system.
\item Bluetooth Low Energy is the underlying technology that makes the
    communication between handheld and wearable possible.
\item Bluetooth sockets were used to work around the one-to-one device
    restriction of Android Wear.
\item AppCompat libraries were used to ensure compatability with older versions
    of Android.
\item Android L specific features were used such as colouring the status bar.
\end{enumerate}

\section{Future Work}

At the time of writing, Android Wear was a brand new technology in it's infancy.
The project started using Android 4.4W and in the space of the short development
window for this project, has already moved up to Android 5.0.2. As a result many
changes will be made to the APIs in a short space of time. Here are some changes
that may be possible in the future or just for further development:

\begin{enumerate}
\item Support for multiple wearable devices to connect to a single phone
\item Support for communicating with the new iteration of Google Glass
\item Support for other Google technologies such as Android TV and Android Auto.
\item WiFi built in to the wearables. Currently many smartwatches use a
    Snapdragon 400 series as their SoC (System on Chip), many of which support
    wireless technologies such as Wi-Fi. At the time of writing, Android Wear
    does not support access to this technology, but will be implementing it in
    a later iteration. This means that much of the code in this project that
    accesses internet resources could be re-written to download the resource
    directly to the wearable isntead of using the handheld as a proxy.
\item More games could be added to the student application. Currently there is
    only one game for the handheld device, and users may want a couple more.
\item the Student Application could be extended to support devices without
    wearables. Currently the application is only available to devices using 
    Android 4.3+. This is because it is Android Wear's minimum requirement.
    However many of the features of the handheld application do not require
    a wearable, and so the mimimum version could be lowered and
    wearable-specific content could be hidden to those users.
\item Other wearable devices may be announced and supported by Android Wear,
    such as smart-shoes, smart-necklace etc. These devices may run Android Wear
    and as a result may be integrated with the current applications. For example
    students could track their walking efficiency to college using smart-shoes.
\end{enumerate}
