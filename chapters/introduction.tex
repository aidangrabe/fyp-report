% INTRODUCTION

As computer components become smaller, cheaper to manufacture and more powerful,
embedded systems are more accessible than ever before. They can be designed to
perform specific functions with a subset of the hardware required for a typical
personal computer. Thus consuming a fraction of the power, and costing a fraction
of the price.

As more and more devices become interconnected with the "Internet of Things",
embedded systems are playing an ever more important role in people's lives,
keeping them connected and allowing them to control different aspects of their
daily activities from one device, such as a smartphone.\\
Today, you can control the lights in your house, have a live home security
stream, control your television or laptop or get alerts about and alter the
temperature in your home all from the screen of your smartphone.

Since these systems are getting smaller and cheaper to produce, the trend is
moving towards wearable technology. Components such as heart rate monitors,
step counters, proximity sensors etc. are all components that can be worn on
the body and the information can be fed to the user's smartphone via Bluetooth,
WiFi or standard wiring.

With these sensors worn on the body or embedded in the user's clothing, the 
smartphone now has a constant feed of updating sensor data with which it can
make important decisions. For example, if a user is jogging while wearing a
heart rate sensor, the smartphone could play a different tone into the
headphones based on the level of the heart rate. The user could configure this
to encourage them to run faster when the heart rate is below a certain
threshold or encourage them to slow down if it is too high.

These types of wearable sensors can be hugely beneficial to users with medical
conditions. For example, a sensor that automatically detects a user's blood
sugar levels and alerting them via their smartphone when the levels are too
low could help avoid potentially dangerous scenarios.

Wearable sensors would also benefit athletes by providing them with data about
their performance and perhaps predict certain injuries before they occur.

Android Wear, as the title may suggest is Google's attempt to create a set of
APIs for dealing with these wearable sensors. Currently it is tailored
specifically for smart watches, but one could assume that future updates might
support a wider range of wearables such as Google Glass or a devices similar to
Nike's Nike+ range of shoes.

Android Wear runs a very similar version of the Android Operating system to
that running on a mobile phone or tablet, but with certain parts stripped away
(eg. Camera, WiFi) or tailored for the smaller screen size. This means
developers can use most of the regular Android APIs and can even send
serialized objects over Bluetooth and have them reconstructed on the other side
of the connection.

\section{Aims and Objectives}

The purpose of this project is to outline some of the capabilities and API
usage for the Android Wear operating system and demonstrate some of these
capabilities in a practical application for college students. The project
consists of three main components:

\begin{enumerate}

\item Research project demonstrating different sensors and their data.
\item Mobile/tablet and smartwatch applications for college students which
    allows users to schedule lectures, track results keep a to-do list etc.
\item A Custom Watchface that displays a user's calendar events along the
    circumference of the watch face.

\end{enumerate}

\section{Document Structure}
\section{Personal Contributions}
