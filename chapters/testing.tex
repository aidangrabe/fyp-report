In this chapter, the testing performed on the Student Application and Calendar
Watchface will be discussed. Testing was performed throughout the development
of the project and no new features were added in the last week before
completion, just testing and error/bug fixing.

\section{Overview}

The following testing was performed on the project:
\begin{enumerate}
\item Unit Testing\\
    Individual units of code tested with as many inputs as possible to ensure
    they work as expected.
\item Performance Testing\\
    Testing the performance of Android Wear and different methods of
    communication to see which methods were more efficient and more responsive.
\item Usability Testing\\
    Day to day testing as well as testing by third parties just by using the
    apps as regular users and logging or reporting bugs when/if they occurred.
\end{enumerate}

\section{Unit Testing}

Since the source code for the project is written in Java, JUnit was used and
easily integrated into the project's workflow. Test cases for individual classes
were written, trying different inputs and edge cases or unlikely cases were
tested to ensure the class could handle the input.

Unit tests can only really be performed on strictly Java code, however testing
can be performed on Android views and state using Android's built-in testing
framework. This framework allows you to test whether views are visible, input
was received and things a unit test might not be able to test.

\section{Performance Testing}
TODO

\section{Usability Testing}
Usability testing was performed by using the application on a day to day basis.
Using the application as a normal user might, and loggin bugs as they appeared.
The project is hosted on Github, which also has a built-in issue tracker. When
a bug appeared during regular use, it would be logged on paper if not near an
internet-facing device, or logged directly into the Github issue tracker where
it could be assigned a category and tag and where it's state would be logged so
it would not be forgotten.

Testing was also performed by third parties such as friends and family. They
would follow the same reporting process by logging issues in the tracker, or
by keeping note of the issues themselves, and passing them on when appropriate.

Due to the nature of a watchface being always visible to the user, it was easy
for people to test as it was always in the foreground, and if something went
wrong, it was immediately noticeable as the watchface might restart or stop
altogether, so the next time the user checked the time, they'd know there was
a problem and could forward on the logs and the stack-trace of the crash.
